\capitulo{7}{Lineas de trabajo futuras}

Este proyecto se concibe como la ampliación de un trabajo previo \cite{saragonz91:online}, avanzando hacia la creación de un prototipo autónomo capaz de trabajar en coordinación con una aplicación software. Se ha logrado el objetivo general mediante el desarrollo de una página web, pero existen aspectos que requieren la continuación del trabajo de mejora. Entre estas mejoras destacan el fortalecimiento de la seguridad, el desarrollo de un diseño web más atractivo y el lanzamiento de la página web para su acceso desde cualquier dispositivo, ya que actualmente únicamente opera en modo local. Además, se debe considerar la posibilidad de mejorar la comunicación en tiempo real mediante la implementación de WebSockets, así como el desarrollo de una aplicación móvil que complemente a la plataforma web.

En cuanto a las funcionalidades del sistema, se propone mejorar el procesamiento de datos y obtener nuevas mediciones con el fin de generar unas estadísticas clínicas más precisas y relevantes. Esto incluye la implementación de un registro de la administración de medicamentos, ofrecer la posibilidad de registrar actividades sin que el Bluetooth esté conectado, y facilitar el proceso de calibración del sensor encargado de la monitorización.

A pesar de que no era un objetivo principal, este proyecto ha logrado una mejora en el prototipo hardware que facilita la realización de pruebas, pero no se han satisfecho necesidades hardware identificadas en la versión anterior. Además, se han detectado nuevos requerimientos. Las posibilidades de trabajo sobre el dispositivo, van desde la implementación de un sistema de alerta para episodios de congelación de la marcha, hasta mejoras en la precisión del sensor MPU6050 y añadir otro adicional para la pierna derecha.

En el Anexo \textit{Instrucciones para la modificación o mejora del proyecto}, se incluyen comentarios de interés y se detallan a fondo cada uno de los aspectos susceptibles de mejora mencionados en este apartado.