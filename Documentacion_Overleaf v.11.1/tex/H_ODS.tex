\apendice{Anexo de sostenibilización curricular}

En el desarrollo de este Trabajo de Fin de Grado (TFG) he tenido la oportunidad de aplicar los principios de sostenibilidad adquiridos durante mi formación académica, profundizando en su comprensión más allá del ámbito teórico. Esta expereiencia me ha permitido entender el impacto y los riesgos que la actividad profesional de un ingeniero de la salud puede tener sobre la sociedad y el medio ambiente. 

A continuación, se describe cómo el proyecto respeta y se adhiere a los principios éticos de justicia social y calidad ambiental.

\begin{itemize}
    \item \textbf{Respeto a los derechos humanos y derechos fundamentales}.\\
    El enfoque del proyecto respeta los derechos humanos y fundamentales, particularmente el derecho a la salud y a gozar de los beneficios derivados del progreso científico. El uso de tecnología de bajo coste y la decisión de emplear hardware y software libre promueven la igualdad de todos los individuos para el acceso a la solución tecnológica obtenida.
    
    \item \textbf{Respeto a la igualdad de género, no discriminación y principios de accesibilidad universal}.\\
    El diseño sencillo e inclusivo de la interfaz de la página web garantiza la accesibilidad a todos los usuarios, independientemente de su género, edad, raza, capacidades físicas y habilidades tecnológicas. Las características del sistema lo convierten en un recurso valioso y accesible para cualquier paciente de Párkinson, contribuyendo a la inclusión. Otro aspecto que promueve la igualdad de acceso es la elección de software y hardware libre en el desarrollo del proyecto.
    
    \item \textbf{Tratamiento de la sostenibilidad y el cambio climático}.\\
    Consciente del impacto ambiental de la tecnología, en el desarrollo se ha querido reducir la huella ecológica en la medida de lo posible. Estas consideraciones condujeron a la elección de una batería recargable y materiales con una larga vida útil.
\end{itemize}

Este anexo, y el proyecto en su conjunto, reflejan mi compromiso con una evolución tecnológica innovadora que responde a las necesidades sociales, de acuerdo con la protección de los derechos humanos y atendiendo a la actual situación de crisis climática.

