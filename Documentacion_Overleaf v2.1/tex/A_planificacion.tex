\apendice{Plan de Proyecto Software}

\section{Introducción}

bla bla bla bla bla bla bla bla bla bla bla bla bla bla bla bla bla bla bla bla bla bla bla bla bla bla bla bla bla bla bla bla bla bla bla bla bla bla bla bla bla bla bla bla bla bla bla bla bla bla bla.


\section{Planificación temporal}


\section{Planificación económica}
En una planificación económica realista hay que considerar todos los costes de realización del proyecto. Esto incluye desde la valoración económica de herramientas de desarrollo software hasta la de materiales, equipos y sueldos del personal necesario.
Además de los costos, también se debe tener en cuenta la rentabilidad a través de diversas opciones de explotación.

En este apartado se presenta la planificación económica del proyecto realizado. Se incluye el desarrollo software de la página web y el montaje hardware de un prototipo anterior sobre el que se han realizado modificaciones.

\subsection{Coste de personal}
Ha habido únicamente una persona involucrada en el proyecto, desde el desarrollo hasta el diseño, que para el cálculo de los costos de contratación será considerada Ingeniero de la Salud sin experiencia. 
En España, un Ingeniero Biomédico recién egresado o con menos de 3 años de experiencia, que es la situación que más se ajusta a este caso, puede aspirar a un salario medio bruto de 29.020 €/año \cite{jobtedIngenieroBiomedico}. Teniendo en cuenta que el contrato necesario para la realización de esté proyecto debe ser de 2 meses de duración a jornada completa, el costo del empleado será de 4.836,66 €, cantidad a la que habrá que añadirle el pago de la Seguridad Social que corre a cargo de la empresa. El cálculo de la cantidad de dinero a pagar según el Régimen General de la Seguridad Social \cite{SeguridaSocial:online} se puede calcular a través de unos porcentajes a aplicar sobre el salario bruto como se muestra en la tabla \ref{tab: costesEmpleado}.

\begin{table}[]
    \centering
    \begin{tabular}{lll}
        \hline
        \rowcolor[HTML]{FFFFFF} 
        \textbf{Sueldo bruto (€/mes)} & 2.418,33 \\ \hline
        \rowcolor[HTML]{EFEFEF} 
        \multicolumn{2}{|c|}{\textbf{Costes del Régimen General de la Seguridad Social}} \\ \hline
        \rowcolor[HTML]{FFFFFF} 
        Contingencias Comunes (23,60\%) & 570,72 \\ \hline
        \rowcolor[HTML]{EFEFEF} 
        Contrato duración determinada Tiempo Completo (6,7\%) & 162,02 \\ \hline
        \rowcolor[HTML]{FFFFFF} 
        FOGASA (0,2\%) & 4,83 \\ \hline
        \rowcolor[HTML]{EFEFEF} 
        Formación Profesional (0,6\%) & 14,51 \\ \hline
        \rowcolor[HTML]{FFFFFF} 
        \textbf{Costo total - 1 mes} & \textbf{3.170,41} \\ \hline
        \rowcolor[HTML]{C0C0C0} 
        \textbf{Costo total - 2 meses} & \textbf{6.340,82 €} \\ \hline
    \end{tabular}
    \caption{Desglose de costes de contratación}
    \label{tab: costesEmpleado}
\end{table}

\subsection{Costes de software}
Todas las herramientas software empleadas compartían la característica de ser de código abierto y gratuitas. Visual Studio Code, XAMPP, Node.js y Arduino IDE han sido algunas de las utilizadas.

\subsection{Amortización de los equipos}
El único equipo indispensable para el desarrollo del proyecto, ha sido un portátil de la marca HP (modelo HP Pavillion Laptop 15-cs2xxx) cuyo precio de compra fue de 849,45€ y al que se le estima una vida útil de 6 años. La amortización \footnote{Amortización es la pérdida de valor de un bien o activo a lo largo del tiempo.} del portatil durante los dos meses y medio que duró el trabajo puede calcularse aplicando la fórmula de la figura \ref{fig:amortizacion-portatil} , obteniendo como resultado 29,49 € \cite{Amortizacion:online}.

\begin{figure}[h]
    \centering
    \[
    \text{Amortización del Periodo} = \frac{\text{Precio de Compra}}{\text{Vida Útil en Años}} \times \frac{\text{Meses de Uso}}{12}
    \]
    \caption{Fórmula para el Cálculo de la Amortización durante un Periodo Determinado}
    \label{fig:amortizacion-portatil}
\end{figure}

\subsection{Costes de hardware}
Se va a estimar el precio del prototipo empleado para la realización de pruebas a lo largo del proyecto. En este caso el cálculo de costos se ha limitado al único dispositivo que todavía requiere mejoras pero, si se obtuviera un producto final susceptible de lanzamiento al mercado, además del cálculo del coste por unidad habría que realizar una estimación de la cantidad necesaria de cada material para la producción del número de dispositivos que se pretenda distribuir.
Tras la búsqueda y comparación de precios de todos los componentes necesarios en las tiendas online de Amazon, turiBOT y AliExpress, se opta por escoger la última de ellas ya que presenta una gran diferencia de precios \cite{AliExpre51:online}. Los resultados se pueden visualizar en la tabla \ref{tab:costesComponentes}.

\begin{table}[]
    \centering
    \begin{tabular}{ll}
        \hline
        \rowcolor[HTML]{FFFFFF} 
        \textbf{Componente} & \textbf{Coste (€)} \\ \hline
        \rowcolor[HTML]{EFEFEF} 
        Microprocesador Arduino UNO R3 & 2,00 \\ \hline
        \rowcolor[HTML]{FFFFFF} 
        Cables & 0,70 \\ \hline
        \rowcolor[HTML]{EFEFEF} 
        Acelerómetro + Giroscopio MPU-6050 & 1,17 \\ \hline
        \rowcolor[HTML]{FFFFFF} 
        Módulo bluetooth HC-05 & 4,00 \\ \hline
        \rowcolor[HTML]{EFEFEF} 
        Display LCD 16x2 & 0,71 \\ \hline
        \rowcolor[HTML]{FFFFFF} 
        Módulo I2C & 0,41 \\ \hline
        \rowcolor[HTML]{EFEFEF} 
        2 pulsadores & 0,50 \\ \hline
        \rowcolor[HTML]{FFFFFF} 
        2 resistencias & 0,50 \\ \hline
        \rowcolor[HTML]{C0C0C0} 
        \textbf{Coste Total} & \textbf{9,99 €} \\ \hline
    \end{tabular}
    \caption{Desglose de costes de los componentes}
    \label{tab:costesComponentes}
\end{table}






\section{Viabilidad legal}

