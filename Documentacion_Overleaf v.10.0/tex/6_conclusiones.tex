\capitulo{6}{Conclusiones}

La ejecución de este proyecto ha finalizado con la obtención de un software innovador para la gestión y el seguimiento de la EP, superando los objetivos iniciales. Este desarrollo ha contribuido a la expansión del proyecto anterior, añadiendo funcionalidades, dotándolo de autonomía y mejorando la experiencia del usuario. 

Por otro lado, el trabajo realizado ofrece un amplio abanico de posibilidades para continuar con futuras implementaciones hacia un sistema integral de monitorización de la EP. Este sistema, que tendría un significativo valor clínico, también podría mejorar notablemente la calidad de vida diara de los pacientes.

\section{Aspectos relevantes.}

Uno de los mayores desafíos de este trabajo era establecer una comunicación inalámbrica eficiente y en tiempo real entre el dispositivo Arduino y el servidor web. Con este propósito se evaluaron diferentes tecnologías de comunicación disponibles para Arduino, optando finalmente por la tecnología Bluetooth, de bajo coste y rendimiento adecuado según los requerimientos del sistema. La implementación de este protocolo de comunicación no fue inmediata y se enfrentaron problemas de inestabilidad en la conexión por incompatibilidades entre los formatos de datos enviados por Bluetooth y los esperados por el archivo que manejaba la conexión.

El desarrollo del sitio web presentó otra serie de desafíos por la necesidad de emplear tecnologías con las que nunca antes se había trabajado.

Cabe mencionar la necesidad de mejoras en la fiabilidad de los datos recogidos por el sensor, un aspecto que ha supuesto complicaciones en la realización de pruebas. Estos datos son poco fiables como consecuencia de una falta de exactitud en los procesos de obtención y manejo implementados en una versión anterior a este proyecto.

Los obstáculos mencionados fueron superados de forma que el trabajo concluyó cumpliendo con todas las expectativas.



