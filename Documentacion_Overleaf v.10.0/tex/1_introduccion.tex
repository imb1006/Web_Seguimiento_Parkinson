\capitulo{1}{Introducción}

Uno de los trastornos neurodegenerativos más presentes a nivel mundial es la enfermedad de Párkinson (EP), cuya manifestación clínica más característica es el deterioro progresivo de la habilidad motora. Los síntomas más comunes son el temblor de reposo, rigidez, bradicinesia (lentitud de movimiento) y alteraciones de la marcha. La detección temprana y el monitoreo continuo son cruciales para personalizar la terapia, obteniendo resultados óptimos y retrasando el avance de la enfermedad. 

Este Trabajo de Fin de Grado (TFG) se enfoca en la obtención de una solución tecnológica que, mediante la combinación de hardware y software, implemente un sistema innovador para el monitoreo y asistencia de pacientes con EP. El proyecto prioriza la accesibilidad económica, empleando dispositivos electrónicos de bajo coste y software de código abierto. Mediante el uso de sensores y tecnologías especializadas, se logra la recolección y análisis en tiempo real de datos sobre actividades de marcha, facilitando el acceso inmediato de los usuarios a dicha información y permitiendo una futura implementación de un sistema de alerta que ayude a los pacientes a superar episodios de bloqueo motor.

Por otro lado, el sistema permitirá almacenar la información recogida para que el profesional de la salud pueda realizar un análisis posterior y obtener conclusiones acerca de la progresión de la enfermedad y la efectividad del tratamiento. Abordando la EP desde una perspectiva tecnológica, se espera contribuir a la innovación tecnológica accesible que demanda el sector sanitario para la mejora de la calidad de vida de personas con EP.

El presente documento constituye la memoria del proyecto, un análisis detallado sobre el proceso de desarrollo de un sistema destinado a mejorar la calidad de vida de los pacientes de Párkinson. Inicia con una clara definición de los objetivos, seguida de los fundamentos teóricos que proporcionan una base para que el lector comprenda la documentación del proyecto, y por un análisis sobre trabajos similares. Incluye una extensa explicación de la metodología y herramientas empleadas para lograr los objetivos iniciales. Al concluir el desarrollo del proyecto, se completaron las secciones de resultados y conclusiones, destinadas a evaluar la efectividad del producto final y los logros del trabajo realizado. Una última sección está destinada a analizar las posibles direcciones de mejora y expansión del proyecto.

En el documento de \textit{Anexos} se desarrollan aspectos de interés que proporcionan una información adicional sobre el desarrollo del proyecto. En el \textit{Apéndice A} se establece el plan general de desarrollo del proyecto, el \textit{Apéndice B} contiene toda la documentación necesaria para que el usuario logre el funcionamiento del sistema. Un manual para desarrolladores se muestra en el \textit{Apéndice C}. En el \textit{Apéndice D} se analizan los datos manejados, mientras que el \textit{Apéndice E} detalla las características del diseño. También se incluye información sobre la especificación de requisitos en el \textit{Apéndice F}, las pruebas para verificar el funcionamiento del sistema en el \textit{Apéndice G} y por último, el \textit{Apéndice H} hace una reflexión sobre la sostenibilidad del proyecto.
 
