\capitulo{2}{Objetivos}

Este proyecto busca proporcionar una herramienta para la mejora del manejo de la EP, tanto desde la perspectiva de profesionales sanitarios como de pacientes, mediante la aplicación de tecnologías avanzadas de recopilación y análisis de datos. La idea surge de la necesidad de ampliación de un trabajo previo centrado en la obtención de un dispositivo destinado a la monitorización, análisis y almacenamiento de los datos de la actividad muscular en personas con EP. 

A continuación, se enumeran y explican de forma detallada los objetivos que se persiguen con la realización de este proyecto.

\subsection{Objetivos generales }
Trabajar en un desarrollo avanzado de las posibilidades que ofrece el dispositivo sensorial de análisis de la marcha de personas con EP con el fin de:
\begin{enumerate}
    \item Mejorar el seguimiento de la EP. El acceso a un registro de actividades que reflejan la situación y evolución del paciente en diferentes momentos, permitirá a los profesionales la toma de decisiones informadas relativas a modificaciones del tratamiento o terapia.
    \item Aumentar la autonomía y comodidad del paciente. Facilitar la integración de la recopilación de datos para estudio de la evolución de la EP en el día a día, al mismo tiempo que se proporciona un apoyo ante situaciones de bloqueos en la marcha.
\end{enumerate}

\subsection{Objetivos de desarrollo web}
Están relacionados con el diseño y la implementación de una interfaz web con funcionalidades específicas para diferentes usuarios. 
\begin{enumerate}
    \item Permitir la gestión de usuarios del sistema. Incluye la realización de las operaciones básicas de creación, modificación y eliminación de usuarios e información relacionada con estos.
    \item Facilitar el acceso a la información personal de pacientes, la realización y registro de una actividad, y la consulta de actividades almacenadas.
    \item Diseñar una interfaz que garantice la accesibilidad. Los usuarios a los que está destinado el producto final de este proyecto tendrán distintos niveles de competencia tecnológica, por lo que un diseño sencillo e intuitivo y la gestión eficaz de la información son de vital importancia para una experiencia de usuario satisfactoria.
\end{enumerate}

\subsection{Objetivos de integración y funcionalidad del sistema}
Lograr la conexión inalámbrica y una comunicación en tiempo real entre la aplicación web, la base de datos con toda la información de usuarios y actividades, y el prototipo de monitorización y detección de bloqueos en la marcha parkinsoniana. Notar que el dispositivo de registro de datos fue desarrollado empleando la plataforma de electrónica Arduino (concretamente Arduino UNO).
\begin{enumerate}
    \item Evaluar las soluciones tecnológicas que permiten una comunicación eficaz y segura entre Arduino y el dispositivo de almacenamiento de la aplicación web.
    \item Establecer una conexión fiable a través de intranet/internet que facilite la observación continua del paciente. Los datos que registra el dispositivo físico de monitoreo de la marcha deben ser enviados en tiempo real a la web para su visualización. 
    \item Desarrollar una base de datos para el almacenamiento y evaluación posterior de los datos recogidos durante la realización de una actividad.
\end{enumerate}














