\documentclass[a4paper,12pt,twoside]{memoir}

% Castellano
\usepackage[spanish,es-tabla]{babel}
\selectlanguage{spanish}
\usepackage[utf8]{inputenc}
\usepackage[T1]{fontenc}
\usepackage{lmodern} % Scalable font
\usepackage{microtype}
\usepackage{placeins}

\RequirePackage{booktabs}
\RequirePackage[table]{xcolor}
\RequirePackage{xtab}
\RequirePackage{multirow}

% Links
\PassOptionsToPackage{hyphens}{url}\usepackage[colorlinks]{hyperref}
\hypersetup{
	allcolors = {red}
}

% Ecuaciones
\usepackage{amsmath}

% Rutas de fichero / paquete
\newcommand{\ruta}[1]{{\sffamily #1}}

% Párrafos
\nonzeroparskip

% Huérfanas y viudas
\widowpenalty100000
\clubpenalty100000

\let\tmp\oddsidemargin
\let\oddsidemargin\evensidemargin
\let\evensidemargin\tmp
\reversemarginpar

% Imágenes

% Comando para insertar una imagen en un lugar concreto.
% Los parámetros son:
% 1 --> Ruta absoluta/relativa de la figura
% 2 --> Texto a pie de figura
% 3 --> Tamaño en tanto por uno relativo al ancho de página
\usepackage{graphicx}

\newcommand{\imagen}[3]{
	\begin{figure}[!h]
		\centering
		\includegraphics[width=#3\textwidth]{#1}
		\caption{#2}\label{fig:#1}
	\end{figure}
	\FloatBarrier
}







\graphicspath{ {./img/} }

% Capítulos
\chapterstyle{bianchi}
\newcommand{\capitulo}[2]{
	\setcounter{chapter}{#1}
	\setcounter{section}{0}
	\setcounter{figure}{0}
	\setcounter{table}{0}
	\chapter*{#2}
	\addcontentsline{toc}{chapter}{#2}
	\markboth{#2}{#2}
}

% Apéndices
\renewcommand{\appendixname}{Apéndice}
\renewcommand*\cftappendixname{\appendixname}

\newcommand{\apendice}[1]{
	%\renewcommand{\thechapter}{A}
	\chapter{#1}
}

\renewcommand*\cftappendixname{\appendixname\ }

% Formato de portada

\makeatletter
\usepackage{xcolor}
\newcommand{\tutor}[1]{\def\@tutor{#1}}
\newcommand{\tutorb}[1]{\def\@tutorb{#1}}

\newcommand{\course}[1]{\def\@course{#1}}
\definecolor{cpardoBox}{HTML}{E6E6FF}
\def\maketitle{
  \null
  \thispagestyle{empty}
  % Cabecera ----------------
\begin{center}
  \noindent\includegraphics[width=\textwidth]{cabeceraSalud}\vspace{1.5cm}%
\end{center}
  
  % Título proyecto y escudo salud ----------------
  \begin{center}
    \begin{minipage}[c][1.5cm][c]{.20\textwidth}
        \includegraphics[width=\textwidth]{escudoSalud.pdf}
    \end{minipage}
  \end{center}
  
  \begin{center}
    \colorbox{cpardoBox}{%
        \begin{minipage}{.8\textwidth}
          \vspace{.5cm}\Large
          \begin{center}
          \textbf{TFG del Grado en Ingeniería de la Salud}\vspace{.6cm}\\
          \textbf{\LARGE\@title{}}
          \end{center}
          \vspace{.2cm}
        \end{minipage}
    }%
  \end{center}
  
    % Datos de alumno, curso y tutores ------------------
  \begin{center}%
  {%
    \noindent\LARGE
    Presentado por \@author{}\\ 
    en Universidad de Burgos\\
    \vspace{0.5cm}
    \noindent\Large
    \@date{}\\
    \vspace{0.5cm}
    %Tutor: \@tutor{}\\ % comenta el que no corresponda
    Tutor: \@tutor{} \\
  }%
  \end{center}%
  \null
  \cleardoublepage
  }
\makeatother

\newcommand{\nombre}{Inés Martos Barbero}
\newcommand{\nombreTutor}{Guirguis Zaki Guirguis Abdelmessih} 
\newcommand{\dni}{09106453V} 

% Datos de portada
\title{Aplicación web para el seguimiento de la actividad de las personas con enfermedad de Parkinson}
\author{\nombre}
\tutor{\nombreTutor}
\date{\today}


\begin{document}

\maketitle


\newpage\null\thispagestyle{empty}\newpage

%%%%%%%%%%%%%%%%%%%%%%%%%%%%%%%%%%%%%%%%%%%%%%%%%%%%%%%%%%%%%%%%%%%%%%%%%%%%%%%%%%%%%%%%
\thispagestyle{empty}


\noindent\includegraphics[width=\textwidth]{cabeceraSalud}\vspace{1cm}

\noindent D. \nombreTutor, profesor del departamento de departamento, área de área.

\noindent Expone:

\noindent Que el alumno D. \nombre, con DNI \dni, ha realizado el Trabajo final de Grado en Ingeniería de la Salud titulado título del trabajo. 

\noindent Y que dicho trabajo ha sido realizado por el alumno bajo la dirección del que suscribe, en virtud de lo cual se autoriza su presentación y defensa.

\begin{center} %\large
En Burgos, {\large \today}
\end{center}

\vfill\vfill\vfill

% Author and supervisor
%\begin{minipage}{0.45\textwidth}
%\begin{flushleft} %\large
%Vº. Bº. del Tutor:\\[2cm]
%D. \nombreTutor
%\end{flushleft}
%\end{minipage}
%\hfill
%\begin{minipage}{0.45\textwidth}
%\begin{flushleft} %\large
%Vº. Bº. del Tutor:\\[2cm]
%D. \nombreTutorb
%\end{flushleft}
%\end{minipage}
%\hfill

\vfill

% para casos con solo un tutor comentar lo anterior
% y descomentar lo siguiente
Vº. Bº. del Tutor:\\[2cm]
D. \nombreTutor


\newpage\null\thispagestyle{empty}\newpage




\frontmatter

% Abstract en castellano
\renewcommand*\abstractname{Resumen}
\begin{abstract}

Los problemas motores característicos de la Enfermedad de Párkinson (EP) afectan significativamente la función de la marcha, provocando episodios de congelación de la marcha en las etapas más críticas. Esto repercute considerablemente en la calidad de vida de las personas con EP.

Los dispositivos de monitorización disponibles para esta enfermedad son caros y escasos, y son aún menos los enfocados en analizar los parámetros de la marcha. La recopilación y análisis de esta información son esenciales para facilitar la toma de decisiones objetivas e informadas por parte de los profesionales sobre la modificación del tratamiento y adaptación de terapias.

Continuando con un proyecto anterior, cuyo objetivo era proporcionar una herramienta de apoyo en el ámbito clínico y de ayuda para pacientes, se han realizado pequeñas mejoras en el hardware del dispositivo utilizado para el registro de datos y se ha desarrollado un software, concretamente un sitio web. Este avance ha permitido el funcionamiento inalámbrico del dispositivo mediante el empleo de Bluetooth para la comunicación con el servidor web. La transmisión de datos se realiza en tiempo real, lo que permite su visualización desde una interfaz simple que también posibilita la gestión de la recogida de datos. La innovación de la plataforma web consiste en permitir tanto a profesionales como a pacientes acceder de forma sencilla a la información más relevante.

\end{abstract}

\renewcommand*\abstractname{Descriptores}
\begin{abstract}
Enfermedad de Párkinson, problemas motores, congelación de la marcha, análisis de la marcha, monitorización, aplicación web, software, datos en tiempo real, comunicación inalámbrica, Bluetooth, innovación tecnológica.
\end{abstract}

\clearpage

% Abstract en inglés
\renewcommand*\abstractname{Abstract}
\begin{abstract}
The characteristic motor problems of Parkinson's Disease (PD) significantly affect gait function, causing freezing of gait episodes in the most critical stages. This considerably impacts the quality of life of people with PD.

The monitoring devices available for this disease are expensive and scarce, and even fewer focus on analyzing gait parameters. The collection and analysis of this information are essential to facilitate objective and informed decision-making by professionals regarding treatment modification and therapy adaptation.

Continuing with a previous project, whose goal was to provide a support tool in the clinical setting and aid for patients, small improvements have been made to the hardware of the device used for data recording, and software has been developed, specifically a website. This advancement has enabled the wireless operation of the device through the use of Bluetooth for communication with the web server. Data transmission occurs in real-time, allowing its visualization from a simple interface that also enables the management of data collection. The innovation of the web platform lies in allowing both professionals and patients to easily access the most relevant information.


\end{abstract}

\renewcommand*\abstractname{Keywords}
\begin{abstract}
Parkinson's Disease, motor problems, freezing of gait, gait analysis, monitoring, website, software, real-time data, wireless communication, Bluetooth, technological innovation.
\end{abstract}

\clearpage

% Indices
\tableofcontents

\clearpage

\listoffigures

Todas las figuras en las que no se indica lo contrario, han sido elaboradas por Inés Martos Barbero, la autora de este trabajo.

\clearpage

\listoftables

Todas las tablas en las que no se indica lo contrario, han sido elaboradas por Inés Martos Barbero, la autora de este trabajo.

\clearpage


\mainmatter
\capitulo{1}{Introducción}


Descripción del contenido del trabajo y de la estructura de la memoria y del resto de materiales entregados.





\capitulo{2}{Objetivos}

Este proyecto busca proporcionar una herramienta para la mejora del manejo de la EP, tanto desde la perspectiva de profesionales sanitarios como de pacientes, mediante la aplicación de tecnologías avanzadas de recopilación y análisis de datos. La idea surge de la necesidad de ampliación de un trabajo previo centrado en la obtención de un dispositivo destinado a la monitorización, análisis y almacenamiento de los datos de la actividad muscular en personas con EP. 

A continuación, se enumeran y explican de forma detallada los objetivos que se persiguen con la realización de este proyecto.

\subsection{Objetivos generales}
Trabajar en un desarrollo avanzado de las posibilidades que ofrece el dispositivo sensorial de análisis de la marcha de personas con EP con el fin de:
\begin{enumerate}
    \item Mejorar el seguimiento de la EP. El acceso a un registro de actividades que reflejan la situación y evolución del paciente en diferentes momentos, permitirá a los profesionales la toma de decisiones informadas relativas a modificaciones del tratamiento o terapia.
    \item Aumentar la autonomía y comodidad del paciente. Facilitar la integración de la recopilación de datos para estudio de la evolución de la EP en el día a día, al mismo tiempo que se proporciona un apoyo ante situaciones de bloqueos en la marcha.
\end{enumerate}

\subsection{Objetivos de desarrollo web}
Están relacionados con el diseño y la implementación de una interfaz web con funcionalidades específicas para diferentes usuarios. 
\begin{enumerate}
    \item Permitir la gestión de usuarios del sistema. Incluye la realización de las operaciones básicas de creación, modificación y eliminación de usuarios e información relacionada con estos.
    \item Facilitar el acceso a la información personal de pacientes, la realización y registro de una actividad, y la consulta de actividades almacenadas.
    \item Diseñar una interfaz que garantice la accesibilidad. Los usuarios a los que está destinado el producto final de este proyecto tendrán distintos niveles de competencia tecnológica, por lo que un diseño sencillo e intuitivo y la gestión eficaz de la información son de vital importancia para una experiencia de usuario satisfactoria.
\end{enumerate}

\subsection{Objetivos de integración y funcionalidad del sistema}
Lograr la conexión inalámbrica y una comunicación en tiempo real entre la aplicación web, la base de datos con toda la información de usuarios y actividades, y el prototipo de monitorización y detección de bloqueos en la marcha parkinsoniana. Notar que el dispositivo de registro de datos fue desarrollado empleando la plataforma de electrónica Arduino (concretamente Arduino UNO).
\begin{enumerate}
    \item Evaluar las soluciones tecnológicas que permiten una comunicación eficaz y segura entre Arduino y el dispositivo de almacenamiento de la aplicación web.
    \item Establecer una conexión fiable a través de intranet/internet que facilite la observación continua del paciente. Los datos que registra el dispositivo físico de monitoreo de la marcha deben ser enviados en tiempo real a la web para su visualización. 
    \item Desarrollar una base de datos para el almacenamiento y evaluación posterior de los datos recogidos durante la realización de una actividad.
\end{enumerate}

\subsection{Objetivos de desarrollo hardware}
Centrado en la mejora y optimización del prototipo hardware empleado en la monitorización del movimiento y recogida de datos para el análisis de la marcha en personas con EP.
\begin{enumerate}
    \item Realizar mejoras en el prototipo inical. Busca perfeccionar el diseño para facilitar su uso en la realización de pruebas de manera más cómoda y minimizando las restricciones de movimiento que puedan afectar al usuario durante su uso.
\end{enumerate}















\capitulo{3}{Conceptos teóricos}


Explicación de los conceptos teóricos básicos necesarios para que cualquier miembro del tribunal pueda entender el trabajo realizado.

Esta sección puede contener el número de subsecciones que sean necesarias.

\section{Sección}

\subsection{Subsección}

\subsubsection{Sub Subsección}

En esta sección y el resto de secciones de la memoria puede ser necesario incluir listas de items.

\begin{itemize}
    \item item1
    \item item2
    \item item3
    \item item4
\end{itemize}

Listas enumeradas.

\begin{enumerate}
    \item item1
    \item item2
    \item item3
\end{enumerate}

Figuras, como la figura \ref{fig:escudo} que aparece en la página \pageref{fig:escudo}. 

Puedes aprender más de las figuras en la dirección \url{https://es.overleaf.com/learn/latex/Inserting_Images}

\begin{figure}[h]
    \centering
    \includegraphics[width=0.25\textwidth]{img/escudoSalud.pdf}
    \caption{Pie de la figura de la figura bla bla bla}
    \label{fig:escudo}
\end{figure}


También se pueden insertar tablas como \ref{tab:my-table}, que ha sido generada con \url{https://www.tablesgenerator.com/}.

\begin{table}[]
\begin{tabular}{lll}
a & b & c \\
1 & 2 & 3 \\
4 & 5 & 6
\end{tabular}
\caption{}
\label{tab:my-table}
\end{table}

Es necesario que todas las figuras y tablas aparezca referenciadas en el texto, como estos ejemplos.

Todos los conceptos teóricos deben de estar correctamente referenciados en la bibliografía. Por ejemplo, aquí estoy citando la página de \LaTeX{} de Wikipedia.




\section{Estado del arte y trabajos relacionados.}
La exploración del contexto actual y el análisis de trabajos e investigaciones previas es fundamental para comprender la relevancia del proyecto en el ámbito de las tecnologías médicas. Esta revisión bibliográfica presenta los avances tecnológicos más recientes y analiza los proyectos e investigaciones similares. Realizada con el propósito de entender la relevancia y el potencial impacto del trabajo, proporciona una base sólida para su desarrollo y lo alinea con las necesidades y desafíos del sector.

\subsection{Revisión de tecnologías}

\subsubsection{Sensor MPU-6050 en salud}
El MPU-6050, uno modelo IMU ampliamente utilizado en aplicaciones portátiles, destaca por su bajo coste, eficiencia energética y alto rendimiento. Su capacidad de rastrear con precisión tanto movimiento rápidos cómo lentos lo hace ideal para su uso en aplicaciones de monitoreo de salud \cite{Jian2016/06}. Los siguientes estudios resaltan la relevancia de este sensor en la mejora de diagnósticos y tratamientos médicos.

\begin{itemize}
    \item ``Automatic Assessments of Parkinsonian Gait with Wearable Sensors for Human Assistive Systems''\footnote{Evaluaciones autmáticas de la marcha parkinsoniana con sensores portátiles para sistemas de asistencia humana}, artículo de Han et al. (2023), presenta un modelo basado en el aprendizaje automático para la evaluación automática de la marcha en pacientes con enfermedad de Parkinson (EP). Utiliza dos sensores MPU-6050 que sitúa en la espinilla para recopilar los datos de la marcha y proporcionar un valor de la Escala Unificada de Calificación de la Enfermedad de Parkinson (UPDRS). Los resultados obtenidos con este modelo proporcionan una precisión del 84,9\% en la clasificación, hasta un 10\% mayor que las clasificaciones realizadas con otros modelos tradicionales. Este es un avance significativo hacia evaluaciones más objetivas y detalladas de la marcha en pacientes con Parkinson \cite{AutomaticAssessments}.
    
    \item El artículo ``A novel sensor-embedded holding device for monitoring upper extremity functions''\footnote{Un novedoso dispositivo de sujeción con sensores incorporados para monitorear las funciones de las extremidades superiores} de Ma et al. (2022) se centra en el desarrollo de un dispositivo de monitoreo para la rehabilitación de las extremidades superiores. En este dispositivo cilíndrico (SEHD\footnote{Sensor-Embedded Holding Device, que se traduce por 'Dispositivo de Sujeción con Sensores Incorporados'}), el MPU-6050 es un sensor clave que permite monitorizar con precisión la funcionalidad de dichas extremidades a través de registros de la destreza manual, la fuerza de agarre, la aceleración y la velocidad angular. En el estudio se pone a prueba el dispositivo desarrollado con diferentes grupos de personas, concluyendo que puede ser utilizado de forma efectiva para evaluar la rehabilitación en pacientes con disfunciones de las extremidades superiores causadas por accidentes cerebrovasculares, traumatismos o envejecimiento \cite{NovelSensor}.

    \item El proyecto NanoStim pretende desarrollar un sistema que permita la electroestimulación en el domicilio del paciente. Dentro de él se enmarca el artículo ``Motion Sensors for Knee Angle Recognition in Muscle Rehabilitation Solutions''\footnote{Sensores de movimiento para el reconocimiento del ángulo de la rodilla en soluciones de rehabilitación muscular} de Franco et al. (2022), en el que se desarrolla un sistema portátil para reconocer el ángulo de la rodilla a través de un sensor MPU-6050 y un microcontrolador ESP32. Para mejorar la precisión del sistema, se implementaron y evaluaron tres filtros de optimización. La aplicación myHealth se desarrolla de forma paralela al dispositivo hardware para mostrar una representación gráfica del movimiento en tiempo real y facilitar a los pacientres la gestión sus sesiones de tratamiento. Para la transmisión de datos, se emplea la tecnología Bluetooth Low Energy (BLE). Además, se llevaron a cabo pruebas con voluntarios en entornos reales para evaluar la efectividad tanto del dispositivo y como de la aplicación \cite{SensorRodilla}.
\end{itemize}


\subsubsection{Bluetooth en la monitorización de la salud}
La tecnología Bluetooth proporciona una conexión segura que, unida a su bajo precio y consumo, la sitúan entre las tecnologías inalámbricas más populares \cite{zubiete2011review} y que mayores facilidades presentan para su integración en el ámbito médico \cite{francosistemas}. Su empleo en la transmisión de señales biomédicas, sin generar interferencia con otras señales \cite{carranza2011patient}, contribuye a la obtención de sistemas de monitoreo continuo. 

La búsqueda bibliográfica sobre el uso del Bluetooth en el sector sanitario ha revelado una gran cantidad de resultados. Entre ellos, destacan numerosos estudios y artículos enfocados en el desarrollo de dispositivos y aplicaciones para la monitorización del estado de pacientes. Muchos de estos proyectos están relacionados directa o indirectamente con la pandemia de la COVID-19, ya sea directamente con dicha enfermedad o con la necesidad de reducir la dependencia del personal sanitario en procesos como rehabilitaciones y otras actividades \cite{leibold2023smartphone}.

\begin{itemize}
    \item En el artículo de Müller et al. (2022) se analiza la rápida evolución y aplicación de las tecnologías mHealth (salud móvil) durante la pandemia de COVID-19. Una de las aplicaciones más relevantes es el desarrollo de aplicaciones nacionales de rastreo de contactos. Dichas apps móviles emplean la tecnología Bluetooth para facilitar el rastreo de contactos, manteniendo el anonimato y privacidad de los usuarios al mismo tiempo que notifican posibles exposiciones con alguien que ha dado positivo en la enfermedad \cite{COVID:doi/10.2196/26041}.
    
    \item ``A Wearable Inertial Measurement Unit for Long-Term Monitoring in the Dependency Care Area''\footnote{Una Unidad de Medición Inercial vestible para el monitoreo a largo plazo en el área de atención a la dependencia.} es un artículo de Rodríguez-Martín et al. que describe el desarrollo de la Unidad de Medición Inercial (IMU) 9x2, diseñada para el análisis del movimiento en personas dependientes o con EP. El dispositivo recopila datos a través de un acelerómetro, un giroscopio y un magnetómetro. Cuenta con clasificadores en línea integrados en el microcontrolador para el análisis de datos, los cuales pueden ser almacenados en una trajeta microSD. Además, incluye un módulo Bluetooth para la transmisión de datos en tiempo real y la recepción de instrucciones de sistemas externos. La IMU 9x2 destaca por su gran autonomía, portabilidad y capacidad de integrarse con otras tecnologías, lo que amplía su aplicabilidad \cite{IMU}.
\end{itemize}

\subsubsection{Plataformas web en el seguimiento de enfermedades}
Debido al fácil acceso a Internet desde diversos dispositivos y las mejoras tecnológicas de los últimos años, las plataformas web para el monitoreo de la salud han ganado importancia y están cada vez más integradas en la vida diaria de la población \cite{field2002telemedicine}. Estas plataformas permiten la interacción con múltiples dispositivos de salud y manejan grandes cantidades de datos de manera eficiente y automatizada. Son, además, económicas y accesibles para la mayor parte de los usuario, ya que no suponen ningún esfuerzo de instalación o configuración y las actualizaciones se realizan de forma automática.

Las características descritas convierten a las aplicaciones web en herramientas útiles en el ámbito de la salud, especialmente en el seguimiento de enfermedades, como se muestra en el siguiente ejemplo.

\begin{itemize}
    \item ePoint.telemed es una plataforma destinada al monitoreo de pacientes con insuficiencia cardíaca crónica (CHF) desde sus domicilios. El acceso a la plataforma se realiza a través de un navegador web con conexión a Internet, lo que facilita el acceso y su manejo, tanto para los pacientes como para los profesionales de la salud. Dicha conexión, permite el uso de la plataforma en una variedad de dispositivos y asegura la transmisión automática de datos a un servidor central en el hospital. ePoint.telemed maneja y recopila datos de dispositivos de salud, como los medidores de presión arterial y bascular, de forma eficaz. Durante el desarrollo, se contó con la participación de profesionales médicos para garantizar el cumplimineto de objetivos y la utilidad futura del proyecto. En la fase de prueba, los pacientes expresaron su satisfacción con el producto. Tras completar el desarrollo y las pruebas, se inició un ensayo controlado aleatorio (RCT) para evaluar la efectividad clínica y la rentabilidad de ePoint.telemed en el norte de Noruega \cite{ePoint}.
\end{itemize}


\subsection{Revisión de proyectos similares}
La búsqueda de información sobre proyectos previos en el mismo ámbito es esencial para entender las consideraciones actuales e identificar brechas y carencias en la investigación disponible. Este análisis permitirá una mejor comprensión de las necesidades de los usuarios, orientando el desarrollo del nuevo proyecto hacia aportes más relevantes y específicos. 

Los descritos a continuación son aquellos proyectos considerados más afines y cuyo análisis puede ser de mayor relevancia en el desarrollo del nuevo proyecto.
\begin{itemize}
    \item El artículo ``A Non-Invasive Medical Device for Parkinson's Patients with Episodes of Freezing of Gait'' de Punin et al. se centra en el desarrollo de un dispositivo económico de monitoreo de la marcha parkinsoniana, con el objetivo de reducir la frecuencia y la duración de episodios de Freezing of Gait\footnote{congelación de la marcha} (FOG) mediante la aplicación de vibración en las extremidades inferiores. El sistema utiliza dos dispositivos, uno para cada tobillo, que incorporan la IMU MPU-6050 para la obtención de los datos de la marcha. Estos datos son procesados mediante un algoritmo basado en la Transformada Wavelet Discreta. \\
    Además del hardware, se diseñó una aplicación móvil que recibe, a través de Bluetooth, información sobre los datos de aceleración registrados por el módulo MPU-6050. La aplicación muestra una representación gráfica de dicha aceleración, permite al usuario comprobar la energía de la señal y controlar manualmente la activación del estímulo vibratorio. Este proyecto proporciona a los pacientes con EP una forma accesible de manejar los episodios de FOG, mejorando su calidad de vida \cite{punin2019noninvasive}.
    
    \item El andador inteligente descrito en el artículo ``ROS-Based Smart Walker with Fuzzy Posture Judgement and Power Assistance''\footnote{Andador inteligente basado en ROS con juicio de postura y asistencia de potencia.} está diseñado asistir a personas mayores o con dificultades motoras, como las que padecen EP. Este dispositivo utiliza un Sistema Operativo Robótico (ROS) y una variedad de sensores junto a un controlador difuso, para controlar automáticamente la velocidad del andador y ayudar al usuario a ajustar su postura en función de las condiciones del entorno. Los principales componentes hardware son un procesador Raspberry Pi 3 B+, que integra tanto WiFi como Bluetooth, y un microcontrolador PIC. Los múltiples sensores utilizados, como el MPU-6050, sirven para monitorear tanto las acciones del usuario como el entorno. Por otro lado, la aplicación móvil asociada al andador, permite a una persona externa recibir información en tiempo real del estado y la ubicación del paciente. También incorpora una base de datos, desarrollada con MySQL, para almacenar los datos recopilados por los sensores, facilitando el seguimiento y evaluación de la salud del paciente \cite{andador}.
    
    \item PAGAS, acrónimo de `Portable and Accurate Gait Analysis System'\footnote{Sistema portátil y preciso de análisis de la marcha}, es un sistema de análisis de la marcha especialmente útil para personas con EP. El dispositivo esta formado por dos sensores ubicados en las plantillas de los zapatos para registrar el apoyo del talón y despegue de los dedos, además de un microcontrolador Arduino BT con ATmega328 que incorpora tecnología Bluetooth. El microcontrolador se encarga de gestionar la recopilación de datos y su transmisión vía Bluetooth, pero el procesamiento de estos datos para la obtención de los parámetros de la marcha se realiza a través de una aplicación Android. Esta aplicación, que presenta una interfaz gráfica de usuario (GUI) muy básica, permite visualizar los parámetros de la marcha en tiempo real, controlar el sistema, y almacenar los resultados para un seguimiento a largo plazo. PAGAS es una solución innovadora y accesible que permite analizar la marcha en diversos entornos, favoreciendo la supervisión continua y la reducción del gasto sanitario \cite{pagas}.
\end{itemize}

\capitulo{4}{Metodología}

\section{Descripción de los datos.}

Como ya se ha mencionado, este proyecto se basa en un Trabajo de Fin de Grado previo \cite{saragonz91:online}, cuyo producto fue un prototipo para registrar los movimientos de la pierna izquierda durante la marcha en pacientes con EP. Los datos se obtenían a través del sensor inercial MPU-6050 y, tras procesarlos, se visualizaban en una pantalla LCD y en el monitor serie de Arduino. Esta información, presentada de forma clara y comprensible, resulta aplicable al estudio de la EP, poniendo el enfoque en la congelación de la marcha. Todo este proceso transcurría exclusivamente en el microcontrolador Arduino UNO. En este dispositivo se cargaba un programa específico que, trabajando con la librería MPU6050, permitía la lectura de los datos del sensor y su análisis en tiempo real, proporcionando información detallada sobre el número de pasos, la velocidad de la marcha y la detección de bloqueos, elementos clave como indicadores de congelación de la marcha.

En el proyecto actual se trabajó para que los datos de actividad, recogidos y procesados del proyecto anterior, se enviaran vía Bluetooth a la página web desarrollada en este proyecto, así como a la base de datos, cuando el usuario así lo decida. En la web, la información puede visualizarse en tiempo real y, si se ha optado por el almacenamiento, de forma posterior junto a una serie de estadísticas básicas.

Ha sido necesaria la creación de una base de datos para almacenar los datos recogidos por Arduino y para gestionar la información específica de la web, como datos de inicio de sesión y las relaciones profesional-paciente. Esta base de datos define las tablas 'actividades', 'pacientes', 'profesional\_paciente' y 'usuarios'.

La comunicación entre las diferentes partes que conforman el proyecto (Arduino, web y base de datos) se maneja a través de dos archivos intermediarios que trabajan de forma coordinada. 
\begin{itemize}
    \item Un script de python, encargado de las interacciones entre Arduino y el servidor web Node.js. Esto es posible gracias a la implementación de la librería SoftwareSerial en el script de Arduino.
    \item Un servidor Node.js que maneja las peticiones para almacenar y recuperar información de la base de datos.
\end{itemize}


En el \textit{Anexo D} se proprociona una explicación detallada de los scripts y ficheros que intervienen en el manejo de datos, incluyendo además un análisis sobre su tratamiento.

 
\section{Técnicas y herramientas.}


\subsection{Metodologías de desarrollo software}
La clasificación de proyectos es un paso vital para la selección del método de desarrollo más eficaz y que mejor se ajusta a las necesidades del proyecto software que se quiere llevar a cabo.

En la Figura \ref{fig:clasificacionProyectos} se presenta una forma simple de clasificación, basada en los conocimientos sobre las necesidades a cubrir y las características de la solución software \cite{pradel2013ingenieria}.

\begin{figure}[h]
    \centering
    \includegraphics[width=1\textwidth]{img/4.TecnicasHerramientas/Clasificacion.png}
    \caption{Clasificación proyectos software \cite{pradel2013ingenieria}}
    \label{fig:clasificacionProyectos}
\end{figure}

El proyecto de desarrollo de una web para el seguimiento de la evolución de la marcha en pacientes con Parkinson se encuadraría dentro del Grupo 1. Dado que los obtjetivos y enfoque están claramente definidos, sería adecuado escoger el método de ciclo de vida cla´sico o en cascada \cite{pradel2013ingenieria}. Esta elección está justicada por la importancia de la sencillez de aplicación sobre la tolerancia al cambio. En otros casos, en los que la flexibilidad cobra más importancia, podría ser más conveniente aplicar métodos iterativos o ágiles.

\subsubsection{Ciclo de vida clásico o en cascada}
Este método de desarrollo software destaca por la sencillez de su aplicación y su forma de organización similar a una cadena de producción \cite{pradel2013ingenieria}.

Las etapas que conforman el desarrollo de un proyecto siguiendo este método son las mostradas en la Figura \ref{} y descritas \cite{pradel2013ingenieria} a continuación:
\begin{enumerate}
    \item Requisitos. Definir qué debe ser el producto a través de la recopilación y documentación de sus funcionalidades.
    \item Análisis y diseño. Definir los puntos de vista externo e interno del producto. Describir los componentes y cómo interaccionan entre sí.
    \item Implementación. Escribir el código de acuerdo con las especificaciones de análisis y diseño. Generar manuales y producto ejecutable.
    \item Pruebas. Comprobar que el producto final cumple con los requisitos.
    \item Mantenimiento. El producto se distribuye a los usuarios y se corrigen los defectos que estos encuentren.
\end{enumerate}

\begin{figure}[h]
    \centering
    \includegraphics[width=0.7\textwidth]{img/4.TecnicasHerramientas/Cascada.png}
    \caption{Ciclo de vida en cascada \cite{pradel2013ingenieria}}
    \label{fig:cicloCascada}
\end{figure}


\subsection{Herramientas Software}



\subsection{Entornos de programación y programas}

\begin{itemize}
    \item Visual Studio Code (v1.84.2). Avanzado editor de código fuente disponible para Windows, macOS y Linux. Destaca por ser ligero y potente, ofreciendo soporte para JavaScript, TypeScript y Node.js, además de numerosas extensiones para otros lenguajes como Python y PHP \cite{VisualSt63:online}. Se puede obtener desde la página oficial \url{https://code.visualstudio.com/}.
    \item XAMPP (v8.2.0). Distribución de Apache que permite iniciar, detener y configurar de forma sencilla los servicios de MariaDB, PHP y Perl que contiene \cite{XAMPPIns2:online}. Descargar el instaldor desde \url{https://www.apachefriends.org/es/download.html} y ejecutar desde el ordenador.
    \item Node.js (v20.10.0). Entorno de ejecución de JavaScript del lado del servidor, es multiplataforma y de código abierto \cite{Nodejs—I6:online}. Instalación desde \url{https://nodejs.org/en} y comprobación a través de al ejecución del comando 'node -v' en el terminal cdm.
    \item Node Package Manager (v10.2.3). Es, por defecto, el sistema de gestión de paquetes de Node.js \cite{Nodejs—I6:online}. Obtención desde las opciones de instalación de Node.js. Se puede comprobar si la instalación ha sido correcta a través de la ejecución del comando 'npm -v' en el terminal cdm.
    \item IDE Arduino (v1.8.19). Entorno de programación de código abierto, sencillo y extensible a través de bibliotecas C++, para programar placas Arduino \cite{Arduino83:online}. Disponible en \url{https://www.arduino.cc/en/software}.
    \item Balsamiq Wireframes (v4.6.5). Herramienta para la creación de prototipos y maquetas de interfaces de forma rápida y clara \cite{Balsamiq0:online}. Hay una versión para escritorio disponible en \url{https://balsamiq.com/wireframes/desktop/}.
    \item Diagramas.net (v13.0.1). Herramienta en linea para la creación de diagramas y gráficos. Cuenta con diferentes opciones para su uso junto a VS Code, GitHub, Dropbox y herramientas similares de trabajo en equipo \cite{drawio51:online}. Accesible desde \url{https://www.drawio.com/}.
    \item Lucidchart (v1.163.3). Software de diagramación online que permite el trabajo individual y en equipo \cite{Lucid79:online}. Existe una versión gratis y otra más completa de pago. Se puede acceder desde \url{https://www.lucidchart.com/pages/es}.
    \item GitHub (v3.11.2). Repositorio en línea de código fuente que facilita el control de versiones y la colaboración en proyectos software mediante el sistema Git \cite{Git44:online}. Acceso desde Internet en el siguiente enlace: \url{https://github.com/}.
    \item GitHub Desktop (v3.3.6). Aplicación de escritorio que facilita el uso de Git y el trabajo con código almacenado en GitHub, mediante una interfaz gráfica de usuario \cite{Comenzar98:online}. Para instalar visita la página \url{https://desktop.github.com/}.

\end{itemize}


\subsubsection{Lenguajes de programación}
Una variedad de lenguajes han sido utilizados para la obtención de la página web:
\begin{itemize}
    \item Con soporte incorporado en VS Code: HTML, JavaScript y CSS.
    \item Con requisitos previos a su empleo en VS Code:
    \begin{itemize}
        \item PHP. Requiere una extensión de servidor PHP. Se ha empleado PHP Intelephense, disponible en las extensiones de VS Code.
        \item Python (v3.12.1). Para que su uso sea posible en VS Code debe ser instalado en el ordenador de trabajo desde \url{https://www.python.org/downloads/}, y añadir la extensión Python de Microsoft.
    \end{itemize}
    \item Sin requisitos adicionales: SQL y Bootstrap.
\end{itemize}

Además, ha sido necesario el uso del lenguaje Arduino. Basado en Wiring y similar a C++ \cite{Arduino83:online}. Se emplea dentro del Arduino IDE para crear el programa que se va a cargar en el microcontrolador Arduino UNO R3.

\subsubsection{Librerías y paquetes}
Librerías de Arduino:
\begin{itemize}
    \item MPU6050. Librería que permite realizar la lectura del MPU-6050 \cite{MPU605015:online}. Desarrollada por Jeff Rowberg y disponible en su repositorio de GitHub.
    \item I2Cdev. Mejora la estabilidad y eficiencia de la comunicación I2C \cite{MPU605015:online}. Se puede descargar como archivo zip desde el repositorio GitHub de Jeff Rowberg para su posterior subida a Arduino IDE.
    \item Wire. Necesaria para el funcionamiento de la librería I2Cdev. Permite la comunicación con dispositivos I2C \cite{WireArdu98:online}. Disponible en Arduino IDE en "Programa/Incluir librería".
    \item SoftwareSerial. Permite la comunicación, en este caso el envío de datos por Bluetooth, a través de pines digitales de la placa Arduino \cite{Software57:online}. Se puede incluir desde Arduino IDE en "Programa/Incluir librería".
    \item LiquidCrystal\_I2C. Requerida para el manejo de la pantalla LCD a través del módulo I2C \cite{LiquidCr9:online}. Posibilidad de instalar directamente en Arduino IDE siguiendo la ruta "Programa/Incluir librería/Administrar bibliotecas".
\end{itemize}

Paquetes Python:

Para la instalación ejecutar 'pip install pyserial requests' en la terminal de VS Code o cdm.
\begin{itemize}
    \item PySerial. Provee de comunicación serial a la aplicación Python , permitiendo la lectura y la escritura de datos \cite{pyserial96:online}.
    \item Requests. Librería HTTP para el envío de solicitudes de forma sencilla \cite{psfreque89:online}.
\end{itemize}

Paquetes Node.js:

Se deben instalar desde el terminal cdm situado en el directorio de mi proyecto node (en este caso en la carpeta 'Arduino Server').
\begin{itemize}
    \item Express. Framework que proporciona los mecanismos necesarios para el desarrollo de aplicaciones web \cite{Express10:online}. Se obtiene a través de la ejecución del comando 'npm install express'.
    \item Body-Parser. Librería que se utiliza con express para trabajar con datos de solicitudes como JSON y datos de formulario. El comando de instalación es 'npm install body\_parser'.
    \item MySQL. Utilizado para el manejo de la lógica que permite conectar la base de datos y el servidor de Node.js. Para proceder a la instalación ejecutar 'npm install mysql'.
    \item CORS. Se emplea con Express y sirve para habilitar el Intercambio de Recursos de Origen Cruzado (CORS) \cite{Intercam91:online} es decir, cuando el archivo html no se encuentra en el directorio de trabajo de Node.js. Instalación a través del comando 'npm install cors'.
\end{itemize}

\subsection{Tecnologías de comunicación}
La selección del método de conexión adecuado es vital para garantizar el óptimo funcionamiento del sistema y satisfacer las necesidades que este pretende abarcar. Para escoger la tecnología adecuada, es importante analizar las características de los principales métodos de comunicación disponibles para Arduino y aplicables a este proyecto: Ethernet, WiFi y Bluetooth.

\subsubsection{Ethernet}
Ethernet es una tecnología de comunicación que utiliza el protocolo Ethernet para permitir la conexión alámbrica de dispositivos electrónicos en una LAN (Local Area Network). También puede ser utilizada en una WAN (Wide Area Network) \cite{¿QuéesEt14:online} \cite{QuéesEth13:online}.

Permite el envío de datos y la conezión a internet de los equipos conectados por cable Ehternet a la red LAN de forma barata, diable, segura y a una muy alta velocidad \cite{¿QuéesEt14:online} \cite{QuéesEth13:online}.
\begin{itemize}
    \item Shield Ethernet W5100 (Figura \ref{fig:moduloEthernet}). \\
    Es un módulo para Arduino en el que se diferencian dos partes:
    \begin{itemize}
        \item Chip W5100. Controlador de Ethernet que, sin necesidad de un Sistema Operativo, permite implementar la comunicación por internet en Arduino a través de la pila de protocolos TCP/IP \cite{Conectar3:online} para el envío de datos. Además, presenta un buffer interno de 16 Kbytes gracias al cual no consumirá memoria del microprocesador Arduino y que, junto con la pila de protocolos, conseguirá liberar a este de tareas \cite{Conectar3:online}.
        \item Lector microSD. Incorpora un lector microSD que puede servir para almacenar ficheros necesarios que le permitan actuar como servidor \cite{Conectar3:online}.
    \end{itemize}
    Ambos elementos comparten el bus SPI, requiriendo especial atención para evitar conflictos durante su uso conjunto \cite{Ethernet91:online}.
    \item Valoración. \\
    La tecnología Ethernet es alámbrica, lo que supone una gran desventaja si se aplica a este proyecto. No permite cumplir los objetivos principales de aumento de la autonomía y comodidad del usuario.
\end{itemize}

\begin{figure}[h]
    \centering
    \includegraphics[width=0.5\textwidth]{img/4.TecnicasHerramientas/Ethernet.png}
    \caption{Shield Ethernet W5100 \cite{Ethernet92:online}}
    \label{fig:moduloEthernet}
\end{figure}

\subsubsection{WiFi}
WiFi es una tecnología de red de área local inalámbrica (WLAN) que, a través de radiofrecuencia \cite{WhatisWi11:online}, proporciona acceso a Internet a todos los dispositivos conectados y al mismo tiempo les permite interaccionar entre sí \cite{¿Quéesla82:online}. Todo el proceso de comunicación está definido por los protocolos del estándar IEE 802.11 \cite{¿Quéesla82:online}\cite{WhatisWi11:online}.

La conexión a Internet se consigue a través de un router inalámbrico que emite una señal dentro de cuyo alcance cualquier dispositivo, también inalámbrico, podrá conectarse \cite{¿Quéesla82:online}. 
\begin{itemize}
    \item Módulo WiFi ESP8266 ESP01 (Figura \ref{fig:moduloWiFi}). \\
    El módulo para Arduino está conformado por la memoria Flash, principal diferencia entre los diferentes módulos ESP8266; y el SoC (System on Chip) ESP8266, que agrupa diversos componentes entre los que destacan un procesador de 32 bits y un chip WiFi que implementa protocolos TCP/IP \cite{ESP8266l93:online}. Además, destaca entre el resto de la familia ESP8266 debido a su reducido tamaño y bajo precio.

    \item Valoración. \\
    Su principal ventaja frente a la conexión Ethernet es que al ser inalámbrica permite mayor comodidad y movilidad, pero esto también se traduce en una reducción de la seguridad y la velocidad, que además presentará mayores fluctuaciones y por tanto una menor fiabilidad \cite{WiFivsEt71:online}. De todos modos, para este proyecto, la conexión inalámbrica se considera una prioridad, por lo que el empleo de tecnología WiFi sería más adecuado que el de Ethernet.
\end{itemize}

\begin{figure}[h]
    \centering
    \includegraphics[width=0.3\textwidth]{img/4.TecnicasHerramientas/WiFi.png}
    \caption{Shield Ethernet W5100 \cite{WiFi1:online}}
    \label{fig:moduloWiFi}
\end{figure}

\subsubsection{Bluetooth}
Bluetooth es una tecnología inalámbrica de corto alcance que utiliza ondas de radio para la transmisión de datos (principalmente paquetes pequeños) y la conexión directa entre dos dispositivos (hosts) sin la necesidad de una infraestructura de red \cite{¿Quéesla6:online}.

El emparejamiento de dos dispositivos mediante bluetooth es un proceso simple que puede llevarse a cabo de forma manual y en ocasiones automática. La conexión se caracteriza por el salto de los dispositivos emparejados entre los diferentes canales en búsqueda de la menor interferencia \cite{¿Cómofun38:online}, lo que se denomina salto en frecuencia y permite un desempeño consistente y de baja latencia.

\begin{itemize}
    \item Módulo Bluetooth HC-05 (Figura \ref{fig:moduloBluetooth}). \\
    Permite la conexión de Arduino con otro dispositivo de forma muy sencilla ya que es similar a utilizar un puerto serie normal. Su principal característica es que actúa como master y server \cite{Conectar13:online}, lo que significa que permite tanto iniciar como recibir comunicaciones. Por este motivo se elige frente al módulo HC-06, que únicamente permite recibir \cite{Conectar13:online}.

    \item Valoración. \\
    Si se compara la conexión bluetooth con la red WiFi una de las mayores diferencias es el alcance máximo, siendo mayor el de la tecnología WiFi. Además, ambas tecnologías son inalámbricas y conectan dispositivos entre sí, pero, mientras que el Bluetooth se emplea preferiblemente para la conexión de periféricos y dispositivos debido a su mayor potencia, la WiFi se utiliza para dar acceso a Internet ya que se caracteriza por una mayor velocidad \cite{Diferenc89:online}.
\end{itemize}

\begin{figure}[h]
    \centering
    \includegraphics[width=0.3\textwidth]{img/4.TecnicasHerramientas/HC-05.png}
    \caption{Shield Ethernet W5100 \cite{HC0536:online}}
    \label{fig:moduloBluetooth}
\end{figure}

\subsubsection{Selección tecnología}
Con el fin de escoger la tecnología más adecuada para cumplir con los requisitos del proyecto, se realizó un análisis en profundidad de cada una de ellas. La elección del Bluetooth se basa en dos razones principales:
\begin{enumerate}
    \item Ventajas de la conexión inalámbrica para la comodidad de uso. Ethernet es la única tecnología que no cumple con este requirimiento.
    \item Priorizar la conexión entre dispositivos. La WiFi, además de permitir conexión inalámbrica entre dispositivos, proporciona a nuestro dispositivo una conexión a Internet innecesaria y que supone complicaciones adicionales. Esto proporciona una ventaja al Bluetooth, que destaca por su simplicidad y eficiencia en la conexión.
\end{enumerate}

\subsection{Herramientas Hardware}
Partiendo del prototipo desarrollado en \cite{saragonz91:online}, se ha desarrollado una versión básica que implementa el envío de datos de forma inalámbrica a través de un módulo Bluetooth. Este prototipo se utilizó exclusivamente durante el desarrollo de la web para la realización de pruebas.

En este apartado se incluye una breve descripción de todos los componentes hardware utilizados.
\begin{itemize}
    \item Microcontrolador Arduino UNO R3 (Figura \ref{fig:arduinoUNO}.\\
    Placa microcontroladora basada en el ATMega328P y que dispone de entradas y salidas para pines tanto digitales como analógicos \cite{ArduinoUNO82:online}. Es sencilla de programar utilizando Arduino IDE y facilita la carga de programas mediante conexión USB.

    \begin{figure}[h]
        \centering
        \includegraphics[width=0.5\textwidth]{img/4.TecnicasHerramientas/ArduinoUNO.png}
        \caption{Microcontrolador Arduino UNO R3. \cite{ArduinoUNO82:online}}
        \label{fig:arduinoUNO}
    \end{figure}
    
    \item Acelerómetro + Giroscopio MPU-6050 (Figura \ref{fig:mpu}).
    Unidad de Medición Inercial (IMU) de muy bajo precio, cuya comunicación se puede realizar mediante bus I2C o por SPI de forma sencilla, y que incluye 6 ejes (3 del giroscopio y 3 del acelerómetro \cite{MPU605015:online}. Una de sus principales ventajas es su reducido coste.
    
    \begin{figure}[h]
        \centering
        \includegraphics[width=0.3\textwidth]{img/4.TecnicasHerramientas/MPU6050.png}
        \caption{Sensor MPU-6050. \cite{MPU6050Amazon396:online}}
        \label{fig:mpu}
    \end{figure}
    
    \item Display LCD 16x2 con interfaz I2C (Figura \ref{fig:lcd}). \\
    En la pantalla lcd se mostrarán datos enviados por el programa de Arduino para que el usuario pueda consultarlos mientras la actividad está en curso, a pesar de que también se mostrarán en la web. La interfaz I2C simplifica el número de conexiones del módulo lcd con Arduino y añade funcionalidades entre las que se encuentran el control del brillo y ajuste del contraste \cite{16x2LCDd27:online}.
    
    \begin{figure}[h]
        \centering
        \includegraphics[width=0.3\textwidth]{img/4.TecnicasHerramientas/LCDI2C.png}
        \caption{Display LCD 16x2 con interfaz I2C. \cite{lcdFoto88:online}}
        \label{fig:lcd}
    \end{figure}

    \item Módulo Bluetooth HC-05 (Figura \ref{fig:moduloBluetooth}). \\
    Proporciona una forma sencilla de conexión para el envío de datos entre Arduino y otro dispositivo, contando con la ventaja tener funcionalidad master y server \cite{Conectar13:online}.
    
    \item 2 Pulsadores (Figura \ref{fig:pulsador}). \\
    Incluidos en el programa Arduino de tal forma que permitan el inicio y finalización de la actividad de recogida de datos.
    
    \begin{figure}[h]
        \centering
        \includegraphics[width=0.2\textwidth]{img/4.TecnicasHerramientas/Pulsador.png}
        \caption{Pulsador. \cite{Pulsador32:online}}
        \label{fig:pulsador}
    \end{figure}
    
    \item 2 Resistencias necesarias para el correcto montaje de los pulsadores.
    
    \item Cables para realizar las conexiones pertinentes.
\end{itemize}

\subsubsection{Mejoras hardware}
En este trabajo se pretendía, además de lograr la comunicación, realizar unas pequeñas mejoras que permitieran el funcionamiento del prototipo de forma autónoma y con la mayor comodidad posible para el paciente. Se realizó un trabajo de soldadura para asegurar las conexiones entre todos los componentes. Los elementos hardware empleados son los que se describen a continuación.
\begin{itemize}
    \item Caja para prototipos (Figura \ref{fig:cajaPrototipos}). \\
    Almacena todo el hardware necesario para el funcionamiento del prototipo, excepto el módulo MPU-6050, que se ubica en el tobillo izquierdo. Fabricada con un plástico resistente que protejerá el microprocesador y las conexiones entre todos los componentes. Tiene perforaciones diseñadas exclusivamente para los siguientes propósitos:
    \begin{itemize}
        \item Visualización del display LCD.
        \item Acceso a pulsadores e interruptor.
        \item Permitir la conexión USB del microprocesador Arduino UNO para la carga de programas.
        \item Sujección de una parte del conector 5 pines.
    \end{itemize}

    \begin{figure}[h]
        \centering
        \includegraphics[width=0.5\textwidth]{img/4.TecnicasHerramientas/CajaPrototipos.png}
        \caption{Caja para prototipos \cite{CajaProt627:online}}
        \label{fig:cajaPrototipos}
    \end{figure}
    
    \item Batería recargable de 9V y conector de batería (Figura \ref{fig:bateria}).\\
    Provee una fuente de alimentación externa que, junto al módulo Bluetooth, permite la autonomía total del prototipo.

    \begin{figure}[h]
        \centering
        \includegraphics[width=0.5\textwidth]{img/4.TecnicasHerramientas/Bateria.jpg}
        \caption{Batería 9V con conector. Fuente propria}
        \label{fig:bateria}
    \end{figure}
    
    \item Interruptor (Figura \ref{fig:interruptor}).\\
    Mecanismo de encendido y apagado, controla el flujo de energía que llega al sistema desde la batería externa.

    \begin{figure}[h]
        \centering
        \includegraphics[width=0.3\textwidth]{img/4.TecnicasHerramientas/Interruptor.png}
        \caption{Interruptor deslizante. \cite{Interruptor42:online}}
        \label{fig:interruptor}
    \end{figure}

    \item Conector 5 pines (Figura \ref{fig:conector5pines}).\\
    Este conector macho y hembra permite la conexión y desconexión del módulo MPU-6050 del sistema. Proporciona dos partes separables: el sensor y la caja de prototipos con el resto del hardware. Es una funcionalidad que facilita el almacenamiento y uso del prototipo. 

    \begin{figure}[h]
        \centering
        \includegraphics[width=0.2\textwidth]{img/4.TecnicasHerramientas/Conector5pines.png}
        \caption{Conector 5 pines. \cite{Conector5P61:online}}
        \label{fig:conector5pines}
    \end{figure}
    
    \item Cable multihilo flexible (Figura \ref{fig:cableMultihilo}).\\
    El cable contiene un mínimo de cuatro hilos y debe ser flexible para asegurar una completa libertad de movimiento al usuario. Conecta el módulo MPU-6050 a un extremo del conector 5 pines.

    \begin{figure}[h]
        \centering
        \includegraphics[width=0.4\textwidth]{img/4.TecnicasHerramientas/CableMultihilo.png}
        \caption{Cable multihilo flexible. \cite{Cable41:online}}
        \label{fig:cableMultihilo}
    \end{figure}
    
    \item Proto Shield (Figura \ref{fig:shield}).\\
    Fácilmente acoplable al microprocesador, permite la construcción de circuitos mediante soldadura para una mayor seguridad en la conexión de los elementos que componen el prototipo.

    \begin{figure}[h]
        \centering
        \includegraphics[width=0.4\textwidth]{img/4.TecnicasHerramientas/Protoshield.png}
        \caption{Proto Shield. \cite{ProtoShi18:online}}
        \label{fig:shield}
    \end{figure}
    
    
\end{itemize}




\capitulo{5}{Resultados}

\section{Resumen de resultados.}

Breve resumen de los resultados. En caso de ser un trabajo muy experimental, los resultados completos pueden aparecer en su anexo correspondiente.

Debería haber una correspondencia entre los objetivos y los resultados explicados en esta sección

\section{Discusión.}

Discusión y análisis de los resultados obtenidos.
\capitulo{6}{Conclusiones}

Todo proyecto debe incluir las conclusiones que se derivan de su desarrollo. Éstas pueden ser de diferente índole, dependiendo de la tipología del proyecto, pero normalmente van a estar presentes un conjunto de conclusiones relacionadas con los resultados del proyecto y un conjunto de conclusiones técnicas. 



\section{Aspectos relevantes.}



-> La exactitud se vae afectada por la poca fiabilidad del registro y rpocesamiento de los datos en el arduino.




Este apartado pretende recoger los aspectos más interesantes del \textbf{desarrollo del proyecto}, comentados por los autores del mismo.

Debe incluir los detalles más relevantes en cada fase del desarrollo, justificando los caminos tomados, especialmente aquellos que no sean triviales. 

Puede ser el lugar más adecuado para documentar los aspectos más interesantes del proyecto y también los resultados negativos obtenidos por soluciones previas a la solución entregada.

Este apartado, debe convertirse en el resumen de la experiencia práctica del proyecto, y por sí mismo justifica que la memoria se convierta en un documento útil, fuente de referencia para los autores, los tutores y futuros alumnos.





\capitulo{7}{Lineas de trabajo futuras}

Este proyecto se concibe como la ampliación de un trabajo previo \cite{saragonz91:online}, avanzando hacia la creación de un prototipo autónomo capaz de trabajar en coordinación con una aplicación software. Se ha logrado el objetivo general mediante el desarrollo de una página web, pero existen aspectos que requieren la continuación del trabajo de mejora. Entre estas mejoras destacan el fortalecimiento de la seguridad, el desarrollo de un diseño web más atractivo y el lanzamiento de la página web para su acceso desde cualquier dispositivo, ya que actualmente únicamente opera en modo local. Además, se debe considerar la posibilidad de mejorar la comunicación en tiempo real mediante la implementación de WebSockets, así como el desarrollo de una aplicación móvil que complemente a la plataforma web.

En cuanto a las funcionalidades del sistema, se propone mejorar el procesamiento de datos y obtener nuevas mediciones con el fin de generar unas estadísticas clínicas más precisas y relevantes. Esto incluye la implementación de un registro de la administración de medicamentos, ofrecer la posibilidad de registrar actividades sin que el Bluetooth esté conectado, y facilitar el proceso de calibración del sensor encargado de la monitorización.

A pesar de que no era un objetivo principal, este proyecto ha logrado una mejora en el prototipo hardware que facilita la realización de pruebas, pero no se han satisfecho necesidades hardware identificadas en la versión anterior. Además, se han detectado nuevos requerimientos. Las posibilidades de trabajo sobre el dispositivo, van desde la implementación de un sistema de alerta para episodios de congelación de la marcha, hasta mejoras en la precisión del sensor MPU6050 y añadir otro adicional para la pierna derecha.

En el Anexo \textit{Instrucciones para la modificación o mejora del proyecto}, se incluyen comentarios de interés y se detallan a fondo cada uno de los aspectos susceptibles de mejora mencionados en este apartado.


\bibliographystyle{apalike}
\bibliography{bibliografia}

\end{document}
