\capitulo{4}{Metodología}

\section{Descripción de los datos.}

Como ya se ha mencionado, este proyecto se basa en un Trabajo de Fin de Grado previo \cite{saragonz91:online}, cuyo producto fue un prototipo para registrar los movimientos de la pierna izquierda durante la marcha en pacientes con EP. Los datos se obtenían através del sensor inercial MPU-6050 y, tras procesarlos, se visualizaban en una pantalla LCD y en el monitor serie de Arduino. Esta información, presentada de forma clara y comprensible, resulta aplicable al estudio de la EP, poniendo el enfoque en la congelación de la marcha. Todo este proceso transcurría exclusivamente en el microcontrolador Arduino UNO. En este dispositivo se cargaba un programa específico que, trabajando con la librería MPU6050, permitía la lectura de los datos del sensor y su análisis en tiempo real, proporcionando información detallada sobre el número de pasos, la velocidad de la marcha y la detección de bloqueos, elementos clave como indicadores de congelación de la marcha.

En el proyecto actual se trabajó para que los datos de actividad, recogidos y procesados del proyecto anterior, se enviaran vía Bluetooth a la página web desarrollada en este proyecto, así como a la base de datos, cuando el usuario así lo decida. En la web, la información puede visualizarse en tiempo real y, si se ha optado por el almacenamiento, de forma posterior junto a una serie de estadísticas básicas.

Ha sido necesaria la creación de una base de datos para almacenar los datos recogidos por Arduino y para gestionar la información específica de la web, como datos de inicio de sesión y las relaciones profesional-paciente. Esta base de datos define las tablas 'actividades', 'pacientes', 'profesional\_paciente' y 'usuarios'.

La comunicación entre las diferentes partes que conforman el proyecto (Arduino, web y base de datos) se maneja a través de dos archivos intermediarios que trabajan de forma coordinada. 
\begin{itemize}
    \item Un script de python, encargado de las interacciones entre Arduino y el servidor web Node.js. Esto es posible gracias a la implementación de la librería SoftwareSerial en el script de Arduino.
    \item Un servidor Node.js que maneja las peticiones para almacenar y recuperar información de la base de datos.
\end{itemize}


En el \textit{Anexo D} se proprociona una explicación detallada de los scripts y ficheros que intervienen en el manejo de datos, incluyendo además un análisis sobre su tratamiento.

 
\section{Técnicas y herramientas.}




