\apendice{Descripción de adquisición y tratamiento de datos}

Los datos de actividad física generados y procesados por el sistema de monitorización son de fuente propia, obtenidos directamente a partir del funcionamiento del programa Arduino, y no extraídos de ninguna base de datos preexistente.

\section{Descripción formal de los datos}

En esta sección se procede a detallar la estructura, procesamiento, manejo, tipo y formato de los datos empleados en la operativa de la aplicación web.

\subsection{Recopilación de datos en el script de Arduino}
\begin{itemize}
    \item \textbf{Datos del sensor MPU6050}. Obtenidos directamente del sensor a través del programa Arduino.
    \begin{itemize}
        \item Valores del acelérometro (ax, ay, az). Datos de tipo entero (`int') correspondientes a las mediciones del acelerómetro en los ejes X, Y, Z con un rango de -2g a +2g. `getAcceleration(\&ax, \&ay, \&az)' es la función encargada de la lectura de valores y la función `contarPasos()' también emplea valores de acelerómetro para contabilizar los pasos del usuario.
        \item Valores del giroscopio (gx, gy, gz). Datos de tipo entero (`int') correspondientes a las mediciones del giroscopio en los ejes X, Y, Z con un rango de -250º/seg a +250º/seg. `getRotation(\&gx, \&gy, \&gz)' es la función encargada de la lectura de valores.
    \end{itemize}
    \item \textbf{Datos procesados de la actividad}. Empleo de variables generadas y administradas por el programa Arduino.
    \begin{itemize}
        \item Variables de actividad. Son de tipo cuantitativo (`float' o `int') e incluyen número de pasos (contP y Ptotal), bloqueos, tiempo de actividad (tiempo\_actividad), velocidad media (velocidadMedia), frecuencia de pasos (frecuencia) y tiempo de actividad en minutos (actividadMin). 
        \item Variables auxiliares. Destinadas al control del tiempo, actividad y cálculo de longitud del paso. De tipo `unsigned long' para variables de tiempo (tiempo1, tiempo2, tiempo), `float' para cont\_espera y cte e `int' para altura y estado.
    \end{itemize}
    
\end{itemize}


\subsection{Datos de comunicación Bluetooth}
El manejo de datos en la comunicación Bluetooth se realiza a diferentes niveles. A continuación, se procede a analizar el proceso.

\subsubsection{Arduino}
La funcionalidad del Arduino se centra en:
\begin{itemize}
    \item \textbf{Recepción de datos}: Arduino está constantemente pendiente de la recepción de comandos a través de Bluetooth, utilizando la función `btSerial.available()'. La información que recibe se lee como cadena de texto (`String') y se almacena en la variable `command' para su procesamiento. Si el Arduino recibe `1' o `0', inicia o detiene la actividad. En caso de recibir información con el formato `altura:<valor>,sexo:<valor>', procede a extraer y procesar los valores.
    \item \textbf{Envío de datos}: Se envían datos cuantitativos (`float' e `int') que están principalemente relacionados la actividad realizada. Además, se incluye información sobre el estado de la actividad para permitir la actualización correspondiente en la plataforma web cuando el control de la actividad se realiza a través de comandos hardware. Para enviar estos datos, se utiliza la función `btSerial.print()'.
\end{itemize}

\subsubsection{Bridge.py}
Este script de python actúa como puente entre el Arduino y el servidor web. Se encarga de:
\begin{itemize}
    \item \textbf{Recibir datos de Arduino}: Mediante la biblioteca `serial' comprueba si se están recibiendo datos en el puerto serie. Según el tipo de datos recibido lleva a cabo diferentes acciones. Al recibir `IZQUIERDA' establece un valor booleano `true', mientras que al recibir otro tipo de datos establece el valor `false' y los procesa para obtener la información en un formato que pueda ser manejado posteriormente.
    \item \textbf{Enviar datos al servidor}: Tras haber recibido datos desde el Arduino y haber realizado el procesamiento correcto, se emplea la biblioteca `requests' para proceder al envio de datos al servidor. Los datos pueden incluir estados de la actividad (`detenida' o `iniciada'), booleanos (flag de izquierda) o datos de actividad procesados como valores numéricos.
    \item \textbf{Enviar comandos a Arduino}: Recibe comandos en formato cadena de texto (`String') desde el servidor web y los transmite a Arduino añadiendo un salto de línea para que este los procese de forma correcta.
\end{itemize}

\subsubsection{Server.js}
Archivo de JavaScript que implementa el servidor web mediante Node.js y provee una API REST para la gestión de la comunicación y los datos en el sistema. Facilita las siguientes funciones:

\begin{enumerate}
    \item \textbf{Recibir datos de \textit{`bridge.py'}}: Recepción de diferentes tipos de datos y almacenamiento en las variables correspondientes. Se reciben datos numéricos (`int' y `float') de la actividad registrada, cadenas de texto (`String') sobre el estado de la actividad, y booleanos almacenados en la variable `izquierda' que hacen referencia a la existencia de bloqueos.
    \item \textbf{Gestionar comandos para Arduino}: Almacenamiento y gestión los comandos en formato cadena de texto que tienen que ser enviados al Arduino a través de \textit{`bridge.py'}. Incluye comandos del estado de la actividad, altura y sexo del usuario.
    \item \textbf{Interacción con la base de datos}: Uso de MySQL para el almacenamiento y recuperación de datos relacionados con la actividad e información personal del paciente.
\end{enumerate}


\subsection{Base de datos: estructura y manejo de datos}
Este apartado está destinado a la descripción de la base de datos `WebParkinson' utilizada por la plataforma. En él, se detalla información relativa a la estructura, los datos almacenados y las relaciones establecidas. Este análisis es fundamental para comprender cómo trabajar con los datos recopilados.

La base de datos está compuesta por varias tablas, cada una de las cuales almacena información específica para cada aspecto relevante del sistema. Las relaciones entre tablas se manejan mediante las claves foráneas.
\begin{itemize}
    \item \textbf{Tabla `usuarios'}: Gestiona los datos de los usuarios del sistema, incluyendo administradores, profesionales y pacientes. Contiene los siguientes campos:
    \begin{itemize}
        \item `id\_usuario' (int). Clave primaria con autoincremento que asigna un identificador único a cada usuario.
        \item `nombre', `apellidos',  `contrasena' (varchar). Información personal y de acceso.
        \item `correo\_electrónico' (varchar). Clave única para asegurar que cada correo electrónico sea único en la base de datos. Necesario para el acceso a la plataforma.
        \item `tipo\_usuario' (enum). Indica el rol del usuario en el sistema (administrador, profesional, paciente).
    \end{itemize}
    \item \textbf{Tabla `pacientes'}: Almacena información específica de los pacientes. Esta información se requiere para el cálculo preciso de datos durante el análisis de la marcha. Contiene los siguientes campos:
    \begin{itemize}
        \item `id\_paciente'. Clave primaria y clave foránea, referenciada desde la tabla `usuarios' para relacionar la información del paciente con su perfil de usuario.
        \item `altura' (int). Altura del paciente en centímetros.
        \item `sexo' (enum). Indica el sexo del paciente (M o F).
    \end{itemize}
    \item \textbf{Tabla `profesional\_paciente'}: Realiza las asignaciones de pacientes a profesionales, facilitando el seguimiento de los pacientes por parte de los profesionales. Contiene los siguientes campos:
    \begin{itemize}
        \item `id\_profesional' (int). Clave primaria y foránea, referenciada desde la tabla `usuarios'.
        \item `id\_paciente' (int). Clave primaria y foránea, referenciada desde la tabla `pacientes'.
    \end{itemize}
    \item \textbf{Tabla `actividades'}: Reúne los datos de las actividades físicas de los pacientes.
    \begin{itemize}
        \item `id\_actividad' (int). Clave primaria con autoincremento que asigna un identificador único a cada actividad.
        \item `id\_paciente' (int). Clave foránea, referenciada desde la tabla `pacientes' para vincular cada actividad a un paciente.
        \item `numero\_bloqueos', `numero\_pasos' (int). Número de bloqueos y total de pasos registrados durante una actividad.
        \item `velocidad\_media' (decimal). Velocidad media de la actividad.
        \item `duracion' (float). Tiempo total de la actividad.
    \end{itemize}
\end{itemize}

Se incluyen inserciones iniciales de usuarios con rol administrador y profesional para facilitar la puesta en marcha de la plataforma.

\section{Descripción clínica de los datos.}

\subsection{Significado clínico de las métricas de actividad}
Monitorizar la actividad física en personas con EP proporciona los datos para una comprensión detallada del estado físico y evolución de la enfermedad en cada paciente.

La marcha en pacientes con EP presenta un aumento en el número de pasos, debido a la disminución de su longitud, y una reducción en la velocidad, lo que se conoce como bradicinesia. Estas características están vinculadas a la rigidez y alteraciones en el control motor típicas  de la EP \cite{diBiase2020}. 

Otro factor relevante son los episodios de bloqueo, consistentes en la incapacidad temporal de mover los pies, que se producen en las etapas más avanzadas de la enfermedad. Estos episodios afectan gravemente a la calidad de vida de los pacientes, ya que suponen un alto riesgo de caídas. Por ello, se requiere un seguimiento exhaustivo de los bloqueos para la aplicación de un tratamiento efectivo \cite{diBiase2020}. 

Las escalas de medición más comunmente utilizadas en el análisis de la marcha de pacientes con EP, son la Unified Parkinson's Disease Rating Scale (UPDRS) y la Movement Disorder Society-Unified Parkinson's Disease Rating Scale (MDS-UPDRS) \cite{diBiase2020}. Esta última es una revisión de la UPDRS e incluye aspectos motores y no motores \cite{Escalaun14:online}.

La aplicación web almacena información relativa al número de pasos, bloqueos y velocidad de la marcha, permitiendo el seguimiento constante de estos indicadores de movilidad, clave para evaluar la evolución de la enfermedad y su impacto en las funciones motoras.


\subsection{Relación con intervenciones terapéuticas}

Los datos recopilados proporcionan una base objetiva para el seguimiento de los síntomas, siendo esenciales en la evaluación clínica, la toma de decisiones terapéuticas y la personalización del tratamiento. Tras una visita a la Asociación Párkinson Burgos, en la que se conversó con profesionales especializados en el manejo de esta enfermedad, se confirmó el impacto positivo de la información manejada en este proyecto a través de los siguientes aspectos:
\begin{itemize}
    \item Evaluación y personalización de la rehabilitación. La monitorización continua permite evaluar la efectividad de las distintas terapias y adaptarlas a las necesidades concretas del paciente. Por ejemplo, ante un aumento de las situaciones de congelación de la marcha puede ser necesario implementar estrategias para prevenir caídas.
    
    \item Ajuste de la medicación. El acceso a información detallada y objetiva sobre la evolución del paciente permitirá a los neurólogos hacer los ajustes precisos en la medicación, basándose en las conclusiones extraídas sobre la eficacia del tratamiento actual.
\end{itemize}


