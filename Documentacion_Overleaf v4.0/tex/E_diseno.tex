\apendice{Manual de especificación de diseño}

\section{Planos}

Si procede

\section{Diseño arquitectónico}
El diseño de la arquitectura web es el primer paso crítico en el diseño de software. Partiendo de los requisitos del sistema, el diseño arquitectónico proporciona una planificación y organización global de este, establenciendo las relaciones entre sus componentes \cite{castro2015arquitectura}.

Este proceso inicia con la obtención de los requisitos funcionales, detallados en el \textit{Anexo B.1}. Sin embargo, el paso esencial para la arquitectura web es el diseño de la experiencia de usuario. En este caso, se emplean diagramas de flujo que muestran las posibilidades del sistema para cada tipo de usuario.

Descripción del diagrama de flujo:
\begin{itemize}
    \item Figura \ref{fig:0_InicioFin}. Muestra el proceso de iniciar sesión en la página web y las opciones que comparten los usuarios en el menú. Redirige a otras figuras diferentes donde el diagrama de flujo continúa según el tipo de usuario que haya iniciado sesión.
    \item Figura \ref{fig:1_Admin}. Recorre las opciones presentadas al administrador en el sistema.
    \item Figura \ref{fig:2_Profesional}. Presenta el flujo de funcionalidades accesibles para el profesional y la relación entre ellas.
    \item Figura \ref{fig:3_Paciente}. Indica la forma de navegación del paciente en la web y la forma de acceder a cada acción posible.
\end{itemize}


\begin{figure}[h]
    \centering
    \includegraphics[width=1\textwidth]{img/E2_DiseñoArquitectonico/0_InicioFin.png}
    \caption{Diagrama de flujo - Inicio y fin de sesión.}
    \label{fig:0_InicioFin}
\end{figure}

\begin{figure}[h]
    \centering
    \includegraphics[width=1\textwidth]{img/E2_DiseñoArquitectonico/1_Admin.png}
    \caption{Diagrama de flujo. Usuario administrador.}
    \label{fig:1_Admin}
\end{figure}

\begin{figure}[h]
    \centering
    \includegraphics[width=1\textwidth]{img/E2_DiseñoArquitectonico/2_Profesional.png}
    \caption{Diagrama de flujo. Usuario profesional.}
    \label{fig:2_Profesional}
\end{figure}

\begin{figure}[h]
    \centering
    \includegraphics[width=1\textwidth]{img/E2_DiseñoArquitectonico/3_Paciente.png}
    \caption{Diagrama de flujo. Usuario paciente.}
    \label{fig:3_Paciente}
\end{figure}


En relación con el diseño de la interfaz de usuario, llevado a cabo mediante los wireframes que se presentan y describen en el \textit{Anexo F.3}, este se asocia principalmente con el desarrollo. Sin embargo, cabe destacar su alineación con la arquitectura web al establecer las pautas para la presentación de los elementos visuales.
